\documentclass[12pt,twocolumn]{article}
\setlength{\columnsep}{.25in}
\pagestyle{headings}

\usepackage{sectsty}
\usepackage[normalem]{ulem}
\allsectionsfont{\sffamily\underline}

\usepackage{amssymb,amsmath}
\usepackage{ifxetex,ifluatex}
\usepackage[utf8]{inputenc}



% Redefine labelwidth for lists; otherwise, the enumerate package will cause
% markers to extend beyond the left margin.
\makeatletter\AtBeginDocument{%
  \renewcommand{\@listi}
    {\setlength{\labelwidth}{4em}}
}\makeatother
\usepackage{enumerate}




\usepackage[unicode=true]{hyperref}
\hypersetup{breaklinks=true, pdfborder={0 0 0}}


\setlength{\parindent}{0pt}
\setlength{\parskip}{6pt plus 2pt minus 1pt}
\setlength{\emergencystretch}{3em}  % prevent overfull lines


\setcounter{secnumdepth}{0}








\begin{document}




\section{A Bitcoin Primer}

A Bitcoin Primer
\begin{itemize}
\item
   -
  Jan--01--2012
\end{itemize}
 by CoinLab.com Authors: Chris Koss, Mike Koss January 1, 2012
What if you could store and transfer money safely, securely, cheaply and
quickly anywhere in the world yourself, without relying on anyone else?

Bitcoin is a new technology that has the potential of supplanting many
of our contemporary banking and money transfer services (at least in the
online economy).

\subsection{What is Bitcoin?}

The term \emph{Bitcoin} refers to both the digital unit of stored value
and the peer-to-peer network of computers transmitting and validating
transactions of these units. The project was publicly
\href{http://bitcoin.org/bitcoin.pdf}{launched in January 2009}, by a
mysterious inventor using the pseudonym ``Satoshi Nakamoto,'' whose
identity is still a mystery. For the first couple of years, it was
mostly just a novelty for computer geeks, hackers, and idealistic
anarchists.

In April 2011, Forbes Magazine's Andy Greenberg wrote an
\href{http://www.forbes.com/forbes/2011/0509/technology-psilocybin-bitcoins-gavin-andresen-crypto-currency.html}{article}
describing the qualities of Bitcoin: it cannot be forged or
double-spent, controlled or inflated by any government, it is not
impeded by international boundaries, has a geek-friendly economy of
\$30,000 per day, and some digital drug-dealers have started accepting
it.

The price of a Bitcoin surged from less than a dollar to over \$30 as a
new demographic became interested: speculators. Geeks who had casually
collected Bitcoin as a curiosity in 2009 found themselves sitting on
tens, or even hundreds, of thousands of dollars. Over the next several
months, Bitcoin prices were extremely volatile, dropping suddenly after
each of a half dozen high profile incidents. Exchanges were hacked,
Bitcoins were lost from carelessness, viruses popped up which stole any
Bitcoin it could find, and some services closed without warning,
disappearing with their customers' money.

As users learned better and safer practices for handling their Bitcoin,
price volatility decreased, and the price of a Bitcoin has climbed to
over \$5. Many new services popped up including margin trading and short
selling, digital downloads, banking and escrow services, a
World-of-Warcraft-style MMORPG where you can gamble on everything with
Bitcoin, web hosting, domain name registration, web design, and currency
exchanges.

\subsection{How does Bitcoin work?}

A Bitcoin \emph{address} is like a bank account, into which a user can
receive, store, and send Bitcoins. Instead of being physically secured
in a vault, Bitcoins are secured with public-key cryptography. Each
address consists of a public key, which is published, and a private key,
which you must keep secret. Anyone can send Bitcoins to any public key,
but only the person with the private key can spend them. While addresses
are public, nobody knows which addresses belong to which people; Bitcoin
addresses are pseudonymous.

The Bitcoin protocol uses
\href{http://blog.ezyang.com/2011/06/the-cryptography-of-bitcoin/}{the
strongest algorithms} used by the NSA for encrypting Secret level
documents. Anyone can generate as many addresses as they want for free.
There are approximately as many possible Bitcoin addresses as there are
atoms in the Earth, so generating duplicate addresses (and thus having
access to someone else's funds) is practically impossible. Most Bitcoin
users maintain a number of addresses, stored in a digital wallet.

When someone wants to send money to another user, they use software
which creates a transaction containing the receiver's address and an
amount, and cryptographically signs it with their private key. This is
published on a peer-to-peer network which validates it against the
sender's public key, checks that the sending address's balance is
sufficient, and propagates it to all the other nodes on the network.

A transaction does not become certified until it is included in a Block
in the Bitcoin Block Chain.

Today, there are thousands of computers \emph{mining} on the Bitcoin
network. Each computer collects transactions broadcast by other nodes
and tries to guess a number which solves an unpredictable cryptographic
problem. A powerful home computer can try 100's of millions of numbers
every second. The more computers that mine, the more difficult finding a
solution becomes; the difficulty is self-adjusting so that, on average,
a new block is found every 10 minutes. The lucky computer that is the
first to find each block earns 50 Bitcoins for its owner.

As each Block is found, it is added to an ever-growing \emph{Block
Chain} (now standing at over
\href{http://btcserv.net/bitcoin/history/}{150,000 Blocks}). Any
transaction listed in the Block Chain is deemed to be valid, and
eliminates the possibility that Bitcoins can be doubly-spent. Since the
only way to re-write history in the Block Chain is to use more computing
power than is available in the rest of the Bitcoin network, it is
generally deemed too costly for any single party to cheat (the raw
computing power of the Bitcoin network is 10 times that of the world's
largest supercomputing center).

The Block Chain allows every Bitcoin client to examine the complete
historical transaction record to determine the current account balance
of every public address in the system.

Since newly created Bitcoins are constantly issued to miners, one would
think that the currency is inherently inflationary (with an ever
expanding money supply). While that is true in the short-term, the rate
of issuing coins is scheduled to be cut in half every four years. So,
while 2.6M Bitcoins are created each year (until January 2013), there
will never be more that
\href{https://en.bitcoin.it/wiki/File:Total\_bitcoins\_over\_time\_graph.png}{21M
total Bitcoins created}. And since Bitcoins are almost infinitely
divisible (up to 8 decimal places), there is no fear that we won't have
enough Bitcoins to deal with an ever expanding economic base of
Bitcoin-denominated transactions.

\subsection{What are the benefits of Bitcoin?}

\subsubsection{Financial Self-Determinism and Control}

The Bitcoin system is unique because it is the first digital store of
value which can be safely and securely saved and transacted by
individuals, without having to rely on a trusted third party. Once
acquired and properly secured, Bitcoins can't be taken from their owner,
by a thief, a bank, or a government. Neither can any entity freeze any
account, nor prevent the owner from performing (essentially free)
transactions on the Bitcoin network.

\subsubsection{Irrevocable Transactions}

Chargebacks are a big problem for many merchants. Virtually all current
payment systems (credit card, inter-bank transfer, PayPal, etc.) allow
the consumer to refute a transaction, and have their funds returned to
them. Merchants have to follow an expensive dispute process to receive
their money and sometimes pay fees of \$10-\$50 per chargeback.
Merchants can be charged additional penalties
\href{http://www.internetretailer.com/2003/09/04/visa-to-lower-fee-inducing-chargeback-ratio-to-1-of-transaction}{up
to \$25,000} if they have an unusually high rate of chargebacks.

Online merchants have chosen to live with a certain amount of fraudulent
chargebacks while expending company resources on various anti-fraud
detection measures. In an effort to minimize chargebacks, merchants
typically ask their customers to reveal personal information about
themselves beyond what is necessary to deliver their product or service,
leading to a loss of personal privacy for the consumer.

Bitcoin transactions reverse the role of trust by being inherently
irrevocable. Once certified in the Block Chain, a transaction cannot be
(practically) reversed. It is incumbent on the consumer to trust each
merchant they are ordering from. Since there are many ways to establish
the credibility of a merchant (e.g., online ratings and word-of-mouth
reputation), the Bitcoin trust system is a good match for Internet
commerce (verifying the trustworthiness of merchants is much easier than
verifying the trustworthiness of all consumers).

Because Bitcoin payments cannot be reversed (without the consent of the
merchant), merchants can offer their products to a wider audience and
require less personal information from their customers.

\subsubsection{No Need for Middlemen}

The policies of payment processors are sometimes not well aligned with
those receiving money; e.g., people who take donations.

In December 2011,
\href{http://www.regretsy.com/2011/12/05/cats-1-kids-0/}{Regretsy}, a
humorous snarky craft blog, raised donations to buy Christmas presents
for children in families undergoing financial hardship. After raising
thousands of dollars, Regretsy's PayPal account was frozen.

When Regretsy's writer, Helen Killer, contacted PayPal support, she was
told that her account was frozen because PayPal's ``Donate'' button can
only be used by non-profit organizations. PayPal later admitted this is
false: any company can use a ``Donate'' button. But PayPal support told
her ``it's not a worthy cause, it's charity,'' and that she would need
to make a new website if she wanted to keep raising money, and that
gifts couldn't be shipped to a different address from the customer who
paid for them (which was odd during the holiday gift-giving season).

By publicizing her frustrating experience on her blog, she eventually
got an apology from PayPal, and they unfroze her account. But there are
many similar stories from other PayPal users who have had accounts
closed or funds frozen. Without an audience to create a public outcry,
many still haven't had their situations remedied.

\href{http://alexking.org/blog/2009/03/23/beware-of-paypal-donation-chargebacks}{Alex
King} is an open source software developer who stopped accepting
donations when some of them started costing him money. In 2009, after an
anonymous user donated \$1 (\$0.67 after PayPal's fees), they charged
back their donation. PayPal then passed a \$10 chargeback fee onto King,
without any prior warning. He says, ``I was never able to issue a refund
to avoid this charge - the refund link was unavailable as the payment
was listed as in dispute.''

PayPal exposes sellers to the risks of frozen accounts and chargeback
fees. The benefit of PayPal, giving customers the ability to get their
money back if they don't receive what they paid for, does not apply in
the donation scenario. Bitcoin transactions are irreversible and can be
accepted without a middle man. As a result, Bitcoin donations can be
accepted without worrying about these risks.

\subsubsection{Low Cost Transactions}

In addition to the unanticipated risks of using payment processors
(e.g., frozen accounts and chargebacks), the known per-transaction costs
of these services can significantly cut into the profits of some
businesses.
\href{https://cms.paypal.com/cgi-bin/marketingweb?cmd=\_render-content\&content\_ID=merchant/merchant\_fees}{PayPal},
\href{https://checkout.google.com/seller/fees.html}{Google Checkout} and
\href{https://payments.amazon.com/sdui/sdui/business/cba#pricing}{Amazon
Checkout's} rates all start at 2.9\% + \$0.30 per transaction,
decreasing to 1.9\% for merchants with over \$30,000 of transactions per
month. Otherwise viable businesses with low profit margins or requiring
many small transactions may not be profitable due to these fees.

Bitcoin transaction fees are voluntary and payments can be accepted
directly by merchants. Assuming a gross profit margin of 20\%,
eliminating processing fees would increase a merchant's profit by 10\%,
as these expenses would come directly off the bottom line.

 \#\#\# A World-Wide System

Unlike current payment processing systems, Bitcoins are inherently
world-wide and multi-national. There are no artificial barriers for
making payments across national boundaries; in fact, it's impossible to
verify a transaction's country of origin. A merchant accepting Bitcoins
immediately has access to a world-wide market, without any risk of
non-payment from those outside his own country's legal enforcement
system.

\subsubsection{An Inflation Hedge for Long-term Savings}

Because the lifetime creation limit is 21M Bitcoins, it may be that they
will be a good way to store long-term value as a hedge against
inflation. This may be especially true for citizens of countries that
are experiencing run-away inflation. If they can transfer their earnings
to Bitcoins, they can be isolated from the rapid inflation of their
native currency, and only convert back when needed to purchase goods or
services using their native currency.

While this strategy is premature due to Bitcoin's very volatile
valuation today, it may become common as Bitcoin becomes more widely
adopted and develops a history of value stability.

\subsection{What are the Inherent Risks of Bitcoins?}

\subsubsection{Irrevocable Transactions}

Merchants do not have to trust their customers to verify payments, but
customers have to now trust merchants to deliver the goods or services
they have paid for. There are methods to alleviate this problem; for
example, use of third-party trusted escrow services which require
merchants to post a performance bond and enter into binding arbitration
of disputes.

\subsubsection{Underlying Value and Volatility in Prices}

What is a Bitcoin worth? The underlying value is a function of the
demand of the currency by consumers, and their ability to use it to
exchange it for other goods and services. Just as fiat currencies no
longer are tied to the value of an underlying commodity, like gold,
Bitcoins are only valuable in as much as people want them and use them.

Numerous public exchanges exist for people to buy and sell Bitcoin in
exchange for dollars or other currencies. This helps establish an
underlying comparative value and allows merchants to cash out of their
Bitcoin holdings on a regular (e.g., daily) basis, minimizing their
exposure to any currency volatility of Bitcoins. While Bitcoins have
fluctuated in value between \$1 and \$30 in 2011 alone, there are
mechanisms for merchants to quote prices in dollar-equivalents (or other
currency), and to exchange the Bitcoins they receive for other
currencies immediately upon receipt.

An additional concern with the price volatility of Bitcoin is that the
total value of all Bitcoins mined so far is just over
\href{http://blockchain.info/stats}{\$30 million}. This relatively small
market cap, in conjunction with a lack of regulatory oversight, exposes
Bitcoin prices to market manipulation.

There is already significant speculation in online forums about who may
be manipulating prices and to what end. When Bitcoin speculators talk
about surprising market movements, they discuss ``The Manipulator,'' a
shadowy individual or group who is manipulating the price of Bitcoin
with their great wealth. Whether they have actually recognized a wealthy
market manipulator or are anthropomorphizing the Invisible Hand of the
market remains unclear.

\subsubsection{Anti-Inflationary}

Noted economist Paul Krugman wrote an
\href{http://krugman.blogs.nytimes.com/2011/09/07/golden-cyberfetters/}{article
in the New York Times} criticizing Bitcoin's anti-inflationary provision
(due to the 21M Bitcoin creation limit). His argument is that Bitcoins
will cause people to hoard the currency rather than spend it. But we
feel his argument ignores the near infinite divisibility of the
currency. If Bitcoin values go up, people will still desire to spend
some of their gains from the currency by using a fraction of what they
own. While fiat currencies are artificially inflated by expanding
government debts, Bitcoin will remain relatively stable in value over
time.

As a creditor, I would be happy to loan Bitcoins as I can be assured
that they won't be artificially inflated before they are returned to me
(with interest).

Contrary to his argument, we also have examples where deflationary
prices in some markets (consumer electronics and computers) would seem
to predict consumers refraining from purchases (why spend \$2,000 on a
computer today when I can wait 2 years and get the same computer for
\$500). Rather, we see a healthy market providing ever-increasing value
to consumers.

\subsubsection{Computational Attack}

The Bitcoin network recognizes the longest Block Chain as the current
valid ledger of all transactions. Block chains can only be extended with
computation-intensive cryptographic hashing. Anyone wanting to
maliciously re-write the history of the Block Chain must have available
greater computational power than the entire remainder of the Bitcoin
network.

Creating this ``alternate history'' does not allow transactions to be
created without a private key, but it has the ability to erase
transactions in the past. Theoretically, a scammer could buy a product
with Bitcoin, and once they receive it, release an alternate block
chain, of greater length than the current one, that does not contain the
scammer's transaction. Because this new block chain is longer, and thus
demonstrates greater past computation, the network will accept it as the
current, most-up-to date block chain. This allows the scammer to spend
Bitcoin to receive a good, then reverse his transaction to keep both the
good and the Bitcoin (i.e., double-spending).

A computational attack would be very difficult to carry out today. The
total computational power of the Bitcoin network is the equivalent of
\href{http://www.bitcoinwatch.com/}{over 100 PetaFLOPs} (the number of
computations it can perform per second). By comparison, this is about 10
times the speed of the world's greatest supercomputer,
\href{http://www.top500.org/lists/2011/11/press-release}{Japan's K
computer}, at 10.51 PetaFLOPs. The expense of creating a large
supercomputer outweighs any potential gains that could come from the
ability to double spend a portion of Bitcoins.

Because of the risk of double spending, it has become common practice in
the Bitcoin community to wait for six confirmations (six ten-minute
blocks to be added to the block chain after your transaction) before
treating a payment as received. While a scammer might get lucky and
reverse one or two blocks with an alternate chain and a great amount of
computation, each additional block is exponentially more unlikely.

\subsubsection{Regulatory Uncertainty}

The legal classification of Bitcoin is still unclear: it could be
considered a commodity, a currency, a financial product, or legally
equivalent to World of Warcraft gold. It remains to be seen what
licenses and financial regulations Bitcoin businesses will be required
to obtain. The largest currency exchange market, MtGox, reportedly has
experienced some
\href{https://bitcointalk.org/index.php?topic=52846.msg635859#msg635859}{difficulties
wiring money} because of money laundering investigations.

Bitcoin is inherently hard to regulate as there is no central authority.
Because transactions are semi-anonymous and accounts cannot be frozen,
it could become a medium of choice for money laundering, tax evasion,
and illicit trade. Using the TOR anonymizing network, any internet user
with some technical savvy can access a service called the Silk Road, a
marketplace for illegal drugs denominated in Bitcoin.

In the above respects, Bitcoin has very similar characteristics to
governmental paper currency, like US dollar bills (i.e., cash). They can
both be transacted nearly anonymously without an easily auditable paper
trail. However, Bitcoin's technological complexity may cause regulators
to view it as a threat to the rule of law. The regulatory classification
and legality of direct party-to-party business transactions are still
uncertain.

\subsubsection{Risk of Loss}

Users of Bitcoin today have to ensure that they secure their digital
wallets from both loss and theft. This can be challenging, requiring use
of secure encryption, password management, and information backup
methods. There have been some high-profile cases where people made
mistakes and lost hundreds of dollars' worth of Bitcoin. With no central
authority to appeal to, these funds are truly unrecoverable.

It is important for Bitcoin adopters to employ best practices and use
methods commensurate with the potential for loss of their Bitcoin
holdings.

\subsubsection{Is Bitcoin ``The One''?}

The Bitcoin system is very young, barely 3 years old. While it has an
engaged community of early adopters, many of whom have done a deep
technical analysis of the security of the Bitcoin protocol, there may be
inherent flaws in the design leading people to abandon the currency in
favor of some other design (or to lose faith in the concept of a
distributed anonymous currency altogether).

Some competing digital currencies have been proposed, but with much more
limited adoption than Bitcoin has seen. It seems likely to us, that
Bitcoin, or something very much like it, will be a viable option for
many types of transactions and exchanges in the online world.

\subsection{Applications Well-suited to Bitcoin}

\begin{enumerate}[1.]
\item
  \textbf{Online sales of digital goods}. Customers can receive delivery
  immediately and the merchant gets a guaranteed irrevocable payment.
\item
  \textbf{Online donations}. Payments can optionally be publicly visible
  to demonstrate social proof of support for a charitable cause.
\item
  \textbf{Super Vault}. A Bitcoin wallet can be created from a
  passphrase or stored on one or more USB-keys. Bitcoins can be
  deposited to the generated public addresses even when the wallet is
  offline. So there is no risk of loss through online hacking; money can
  flow in, but is impossible to flow out without retrieving the offline
  wallet from storage (or the memory of the wallet creator).
\item
  \textbf{\href{http://en.wikipedia.org/wiki/Remittance}{Remittances}}.
  Inexpensive money transfer system across national boundaries. Agents
  could accept cash in a developed country, and transfer Bitcoins to an
  agent in the home country of a foreign worker, to be picked up by the
  family of the worker.
\end{enumerate}
\subsection{References and Links}

\begin{enumerate}[1.]
\item
  \href{http://bitcoin.org/bitcoin.pdf}{Bitcoin: A Peer-to-Peer
  Electronic Cash System} - by Satoshi Nakamoto (original paper)
\item
  \href{http://en.wikipedia.org/wiki/Bitcoin}{Bitcoin} on Wikipedia
\item
  \href{http://www.weusecoins.com/}{We Use Coins} - An Excellent
  introductory video.
\item
  \href{https://bitcointalk.org/}{Bitcoin Forum} - Online discussions of
  Bitcoin by early adopters and enthusiasts.
\item
  \href{https://en.bitcoin.it/wiki/Main\_Page}{Bitcoin Wiki} - Technical
  information on the Bitcoin protocol, software, and services.
\item
  \href{http://bitcoin.org/}{Bitcoin.org} - Primary download site for
  the ``official'' Bitcoin client
  \href{https://github.com/bitcoin/bitcoin}{(source code)}
\item
  \href{http://blockchain.info/}{BlockChain} and
  \href{http://blockexplorer.com/}{Block Explorer} - Online browsers of
  Bitcoin published transactions
\item
  \href{https://mtgox.com/}{MtGox} - The largest Bitcoin exchange
  (Dollars exchanged with Bitcoin) - live price and order book chart at
  \href{http://mtgoxlive.com/orders}{MtGoxLive}.
\item
  \href{https://bitcoinica.com/trading}{Bitcoinica} - The 2nd most
  popular Bitcoin trading site, offers margin and short-selling not
  offered on MyGox.
\item
  \href{https://www.tradehill.com/}{TradeHill} - Another popular
  (international) Bitcoin exchange.
\item
  \href{https://strongcoin.com/}{StrongCoin} - An easy-to-use online
  digital wallet.
\item
  \href{https://www.instawallet.org/}{InstaWallet} - On-demand online
  wallet with no account needed - creates a private URL per address.
\item
  \href{https://deepbit.net/}{DeepBit} - One of the largest mining pools
  for Bitcoin with a combined compute power of 3,000 Giga-hashes per
  second (3 x 10\^{}12 hashes/sec)
\end{enumerate}




\end{document}
